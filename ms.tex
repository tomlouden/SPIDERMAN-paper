% mnras_template.tex
%
% LaTeX template for creating an MNRAS paper
%
% v3.0 released 14 May 2015
% (version numbers match those of mnras.cls)
%
% Copyright (C) Royal Astronomical Society 2015
% Authors:
% Keith T. Smith (Royal Astronomical Society)

% Change log
%
% v3.0 May 2015
%    Renamed to match the new package name
%    Version number matches mnras.cls
%    A few minor tweaks to wording
% v1.0 September 2013
%    Beta testing only - never publicly released
%    First version: a simple (ish) template for creating an MNRAS paper

%%%%%%%%%%%%%%%%%%%%%%%%%%%%%%%%%%%%%%%%%%%%%%%%%%
% Basic setup. Most papers should leave these options alone.
\documentclass[a4paper,fleqn,usenatbib]{mnras}

% MNRAS is set in Times font. If you don't have this installed (most LaTeX
% installations will be fine) or prefer the old Computer Modern fonts, comment
% out the following line
\usepackage{newtxtext,newtxmath}
% Depending on your LaTeX fonts installation, you might get better results with one of these:
%\usepackage{mathptmx}
%\usepackage{txfonts}

% Use vector fonts, so it zooms properly in on-screen viewing software
% Don't change these lines unless you know what you are doing
\usepackage[T1]{fontenc}
\usepackage{ae,aecompl}

%%%%% AUTHORS - PLACE YOUR OWN PACKAGES HERE %%%%%

% Only include extra packages if you really need them. Common packages are:
\usepackage[dvipdfmx]{graphicx}	% Including figure files
\usepackage{amsmath}	% Advanced maths commands
\usepackage{amssymb}	% Extra maths symbols
\usepackage{multicol}
\usepackage{siunitx}
\usepackage{bmpsize}
\usepackage[anythingbreaks]{breakurl}
\usepackage{subfig}
\usepackage{fancyvrb}
\usepackage{hyperref}


\def\startdata{\if@table@not@headed\kill\caption{\\%
    \@tablecaption}\endhead\hline\endfoot%
  \fi%
}

\def\enddata{% 
 \crcr 
 \noalign{\vskip .7ex}% 
 \before@enddata 
 \endtabular 
 \setbox\pt@box\lastbox 
 \pt@width\wd\pt@box\box\pt@box 
}% 


\newcommand{\aprx}{\raise.17ex\hbox{$\scriptstyle\sim$}}

%%%%%%%%%%%%%%%%%%%%%%%%%%%%%%%%%%%%%%%%%%%%%%%%%%

%%%%% AUTHORS - PLACE YOUR OWN COMMANDS HERE %%%%%

% Please keep new commands to a minimum, and use \newcommand not \def to avoid
% overwriting existing commands. Example:
%\newcommand{\pcm}{\,cm$^{-2}$}	% per cm-squared

%%%%%%%%%%%%%%%%%%%%%%%%%%%%%%%%%%%%%%%%%%%%%%%%%%

%%%%%%%%%%%%%%%%%%% TITLE PAGE %%%%%%%%%%%%%%%%%%%

% Title of the paper, and the short title which is used in the headers.
% Keep the title short and informative.
\title[spiderman]{\textsc{spiderman}: an open source code to model phase curves and secondary eclipses.}

% The list of authors, and the short list which is used in the headers.
% If you need two or more lines of authors, add an extra line using \newauthor
\author[T. Louden, L. Kreidberg]{Tom Louden$^{1}$\thanks{E-mail: t.m.louden@warwick.ac.uk} and Laura Kreidberg$^{2}$\\
$^{1}$Department of Physics, University of Warwick, Coventry, CV4 7AL, UK\\
$^{2}$Department of Astronomy, Harvard, America, America, America}

% These dates will be filled out by the publisher
\date{Accepted XXX. Received YYY; in original form ZZZ}

% Enter the current year, for the copyright statements etc.
\pubyear{2016}

% Don't change these lines
\begin{document}
\label{firstpage}
\pagerange{\pageref{firstpage}--\pageref{lastpage}}
\maketitle

% Abstract of the paper
\begin{abstract}

Presenting \textsc{spiderman}, a fast code for calculating exoplanet phase curves and secondary eclipses with arbitrary surface brightness distributions.

The development version of the code is available at \url{https://github.com/tomlouden/spiderman}.

\end{abstract}

\textsc{spiderman}

% Select between one and six entries from the list of approved keywords.
% Don't make up new ones.
\begin{keywords}
planets and satellites: individual (WASP 52b)---stars: individual (WASP 52)---techniques: spectroscopic---planets and satellites: atmospheres---celestial mechanics---atmospheric effects
\end{keywords}

%%%%%%%%%%%%%%%%%%%%%%%%%%%%%%%%%%%%%%%%%%%%%%%%%%

%%%%%%%%%%%%%%%%% BODY OF PAPER %%%%%%%%%%%%%%%%%%

\section{Introduction}\label{sec:introduction}

BATMAN \citep{Kreidberg2015a}

\section{Method}\label{sec:method}

\begin{figure}
\begin{center}
\includegraphics[width=\columnwidth]{img/zhang_quad.pdf}
\caption{This is a caption}
\label{fig:zhang_quad}
\end{center}
\end{figure}

\begin{figure}
\begin{center}
\includegraphics[width=\columnwidth]{img/zhang_quad_bright.pdf}
\caption{Same as \ref{fig:zhang_quad}, but the scale is now flux instead of temperature. A blackbody spectrum and a channel of 1.1-1.7 microns was assumed to make the conversion. The option exists to add limb darkening to this profile.}
\label{fig:zhang_quad2}
\end{center}
\end{figure}

\begin{figure}
\begin{center}
\includegraphics[width=\columnwidth]{img/sphere_quad.pdf}
\includegraphics[width=\columnwidth]{img/spherical_lc.pdf}
\caption{A model generated using spherical harmonics}
\label{fig:harmonics}
\end{center}
\end{figure}

\begin{figure}
\begin{center}
\includegraphics[width=\columnwidth]{img/zhang_lc.pdf}
\caption{This is a caption}
\label{fig:extract_region}
\end{center}
\end{figure}

\begin{figure}
\begin{center}
\includegraphics[width=\columnwidth]{img/zhang_lc_zoom.pdf}
\caption{This is a caption}
\label{fig:extract_region}
\end{center}
\end{figure}

\begin{figure}
\begin{center}
\includegraphics[width=\columnwidth]{img/free_parameterstriangle.pdf}
\caption{This is a caption}
\label{fig:extract_region}
\end{center}
\end{figure}

\begin{figure*}
\begin{center}
\includegraphics[width=\columnwidth]{img/free_parametersflux_map.pdf}
\includegraphics[width=\columnwidth]{img/free_parametersflux_errs.pdf}
\caption{left: right: the constraints on the brightness distribution.}
\label{fig:extract_region}
\end{center}
\end{figure*}

\begin{figure*}
\begin{center}
\includegraphics[width=\columnwidth]{img/free_parameterstemp_map.pdf}
\includegraphics[width=\columnwidth]{img/free_parameterstemp_errs.pdf}
\caption{The underlying temperature distributions generated by the zhang model.}
\label{fig:extract_region}
\end{center}
\end{figure*}

\subsection{Fitting procedure}\label{sec:fitting}

\subsection{Integrator validation}\label{sec:numerical}

\begin{figure}
\begin{center}
\includegraphics[width=\columnwidth]{img/precision.pdf}
\caption{This is a caption}
\label{fig:extract_region}
\end{center}
\end{figure}

\subsection{Performance}\label{sec:performance}

In typical usage, SPIDERMAN is capable of producing over 1000 models per second on a single core. This means that a 1 million MCMC sample can be generated in approximately a quarter of an hour.

The performance tests were carried out on a single core of an Intel Core I5-3470 Processor.

\begin{figure}
\begin{center}
\includegraphics[width=\columnwidth]{img/exec_time.pdf}
\caption{This is a caption}
\label{fig:exec_time}
\end{center}
\end{figure}

\section{The \textsc{spiderman} package}\label{sec:package}

\textsc{spiderman} is an open source project and is being actively developed on github. The code is available on PyPi, and the latest stable version can easily by installed by simply running the command 

\begin{Verbatim}[frame=single]
> pip install spiderman-package
\end{Verbatim}

Or, to access the bleeding-edge development version from github:

\begin{Verbatim}[frame=single]
> git clone https://github.com/tomlouden/SPIDERMAN
> cd SPIDERMAN
> sudo python setup.py install
\end{Verbatim}

Full installation instructions and other information are available in the documentation, at \url{http://spiderman.readthedocs.io}

\begin{figure}
\begin{center}
\includegraphics[width=\columnwidth]{img/system.pdf}
\caption{This is a caption}
\label{fig:extract_region}
\end{center}
\end{figure}

\section{Application to data}\label{sec:Observations}

As a test of the performance of the code the model was applied to a phase curve and secondary eclipse of WASP-43b taken with HST WFC3. These data had previously been published in \citet{Stevenson2014c}. However, in this paper the phase curve and the secondary eclipse are treated as seperate phenomena, when of course, they are both manifestations of the same underlying brightness distribution. Indeed, the bright dayside of the planet is the main driver of the phase curve, and is also the part occulted by the stellar disc. It is therefore more parsimonious, if at all possible, to use a model that accounts for both of these observations simultaneously. This also reduces the total number of paramters needed, which is satisfying from an information content angle.

One could argue that planetary limb darkening, among other effects, will act to decouple the secondary eclipse from the phase curve, and there are not good constraints on what to expect from the limb darkening as it requires a full atmosphere model treatment. However, one would expect that the. \textsc{spiderman} does have the option to include planetary limb darkening, which would allow one to marginalise over the uncertainty in these parameters.

In princple, constraints being placed on the limb darkening can constrain the vertical temperature profile of the planet.

\section{Discussion}\label{sec:Discussion}

\subsection{Future work}\label{sec:future work}

\textsc{spiderman} is still under development, and there are a number of features that are planned to be implemented:

- More brightness distributions, such as orange segments
- Better treatment of errors in the 

\section{Conclusions}

\section*{Acknowledgements}

The majority of this work was carried out at the kavli summer program in physics 2016. T.L extends his gratitude to the program organisers, in particular Jonothan Fortney and Pascale Geurard.

T.L. is supported by a STFC studentship. P.W. is supported by a STFC consolidated grant (ST/L000733/1).

%%%%%%%%%%%%%%%%%%%%%%%%%%%%%%%%%%%%%%%%%%%%%%%%%%

%%%%%%%%%%%%%%%%%%%% REFERENCES %%%%%%%%%%%%%%%%%%

% The best way to enter references is to use BibTeX:

\bibliographystyle{mnras}
%\bibliography{/home/astro/phrmat/Documents/BibTeX/Papers-LowEUV}
\bibliography{bibliography}
%\bibliography{example} % if your bibtex file is called example.bib


% Alternatively you could enter them by hand, like this:
% This method is tedious and prone to error if you have lots of references
%\begin{thebibliography}{99}
%\bibitem[\protect\citeauthoryear{Author}{2012}]{Author2012}
%Author A.~N., 2013, Journal of Improbable Astronomy, 1, 1
%\bibitem[\protect\citeauthoryear{Others}{2013}]{Others2013}
%Others S., 2012, Journal of Interesting Stuff, 17, 198
%\end{thebibliography}

%%%%%%%%%%%%%%%%%%%%%%%%%%%%%%%%%%%%%%%%%%%%%%%%%%

%%%%%%%%%%%%%%%%% APPENDICES %%%%%%%%%%%%%%%%%%%%%

%\appendix

%\section{Some extra material}

%If you want to present additional material which would interrupt the flow of %the main paper,
%it can be placed in an Appendix which appears after the list of references.

%%%%%%%%%%%%%%%%%%%%%%%%%%%%%%%%%%%%%%%%%%%%%%%%%%


% Don't change these lines
\bsp	% typesetting comment
\label{lastpage}
\end{document}

% End of mnras_template.tex